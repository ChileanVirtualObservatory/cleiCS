\section{Introducción}

\subsection{Big Data en Astronomía}

En los últimos años, los ámbitos empresarial, científicos y de la administración
han estado haciendo frente a la avalancha de datos con un nuevo concepto que ha
definido como Big Data. Por la simple denominación usada se entiende que se trata
de grandes volúmenes de información que no es sencillo tratar con las herramientas
y procedimientos tradicionales. Encierra esta idea el tratamiento de información
que hizo evolucionar los métodos y recursos habituales para hacerse cargo de grandes
volúmenes de datos (de terabytes a zetabytes). Estos se generan a gran
velocidad y además se añade una posible componente de complejidad y variabilidad en
el formato de esos datos. Las instalaciones astronómicas de última generación
existentes y las que están en construcción, como el Atacama Large
Millimeter/submillimeter Array(ALMA), Large Synoptic Survey Telescope (LSST), y el
Square Kilometer Array (SKA), producirán datos de gran escala que se proyecta que
en el año 2020 generarán más de 60 Petabytes de datos accesibles a los astrónomos.

Todo ello requiere de técnicas y tecnologías específicas para su captura,
almacenamiento, distribución, gestión y análisis de la información.

Big Data no es una tecnología en sí misma, sino más bien un planteamiento de
trabajo para la obtención de valor y beneficios de los grandes volúmenes de
datos que se están generando hoy en día. Se deben contemplar aspectos como:

\begin{itemize}
    \item Cómo capturar, gestionar y explotar
    \item Cómo asegurar, verificar validez y fiabilidad.
    \item Cómo compartir para obtener mejoras y beneficios.
    \item Cómo comunicar para facilitar la toma de decisión y posteriores análisis.
\end{itemize}

\subsection{Observatorios en Chile}

Las privilegiadas condiciones atmosféricas hacen de Chile uno de los lugares más
propicios para la realización de investigaciones científicas en astronomía.

Existen más de una docena de instalaciones astronómicas de gran envergadura a lo
largo de nuestro territorio nacional \cite{observatorios_chile}, como por ejemplo
``Atacama Large Milimeter/submilimeter Array'' (ALMA), ``Very Large Telescope''
(VLT), y en los próximos ``European Extremely Large Telescope'' (E-ELT), con el cual
se estima que el 60\% de la observación astronómica mundial se realice en Chile.
Una de las condiciones que se establecen a nivel país, es que el 10\% del tiempo de
observación pertenece a la comunidad astronómica chilena.
Estos generan datos a gran escala, justificando a nivel país, el desarrollo de una
plataforma astroinformática para su administración y análisis inteligente.

Esto como necesidad no es nuevo, desde el 2002 se pensó en este tipo de problemática
y una forma de abordarlo fue creando el Observatorio Virtual (VO).
El VO es una iniciativa internacional que permite el acceso a archivos astronómicos
y centros de datos a astrónomos y personas comunes a través de Internet.
Con la estandarización de métodos e información es posible estudiar los registros
astronómicos sin requerimientos físicos de instrumentos y locación.

\subsection{¿Por qué es nuevo?}
VO con datos de ALMA.

Resumen de lo importante del paper.

%\subsection{IVOA Standards}


\subsection{ALMA + Cubos + Formatos propios}
