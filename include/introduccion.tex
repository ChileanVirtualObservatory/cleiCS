\section{Introducción}

\subsection{``Big Data'' en Astronomía}

En los últimos años, los ámbitos empresarial, científicos y de administración
han estado haciendo frente a la avalancha de datos proveniente de distintas
fuentes asociadas a cada investigación, concepto que se ha definido como
\emph{Big Data}.
Por su simple denominación, se entiende que trata de la manipulación de grandes
volúmenes de información, los cuales no son sencillos de procesar con las herramientas
y procedimientos tradicionales.
Con la idea de procesamiento de información a gran escala,
la evolución de los métodos y recursos habitualmente utilizados
ha sido la responsable de ser capaces de manipular grandes volúmenes de datos,
los cuales pueden llegar a ser del orden de TeraBytes a ZetaBytes.
Adicionalmente, no sólo estamos tratando con grandes volúmenes de datos
de manera estacionaria, si no que la frecuencia con la cual éstos son generados,
crea nuevos componentes críticos para las el desarrollo de soluciones,
como lo son el almacenamiento, variabilidad de formato, y tiempo de respuesta.
Es en el área de la astronomía, donde podemos ver el concepto de \emph{Big Data}
en una situación real, en las cuales las instalaciones de última generación
recientes, como el Atacama Large Millimeter/submillimeter (ALMA),
y las que estan en construcción, como el Large Synoptic Survey Telescope (LSST) y el
Square Kilometer Array (SKA)
producen y producirán datos de gran escala, proyectándose que para el año 2020
serán más de 60 PetaBytes de información accesible para la comunidad astronómica.

Considerando todos los aspectos técnicos y tecnológicos de la manipulación
de grandes volúmenes de datos en los ejemplos anteriormente señalados, requieren
el desarrollo de sistemas específicos para cada uno de ellos,
los cuales consideran aspectos de captura, almacenamiento, distribución, gestión
y análisis de la información.

\emph{Big Data} no es una tecnología en sí misma, sino más bien un planteamiento de
trabajo para la obtención de valor y beneficios de los grandes volúmenes de
datos que se están generando hoy en día. Se deben contemplar aspectos como:

\begin{itemize}
    \item Cómo capturar, gestionar y explotar
    \item Cómo asegurar, verificar validez y fiabilidad.
    \item Cómo compartir para obtener mejoras y beneficios.
    \item Cómo comunicar para facilitar la toma de decisión y posteriores análisis.
\end{itemize}

\subsection{Observatorios en Chile}

Las privilegiadas condiciones atmosféricas hacen de Chile uno de los lugares más
propicios para la realización de investigaciones científicas en astronomía.

Existen más de una docena de instalaciones astronómicas de gran envergadura a lo
largo de nuestro territorio nacional \cite{observatorios_chile}, como por ejemplo,
el anteriormente nombrado ``Atacama Large Milimeter/submilimeter Array'' (ALMA),
el ``Very Large Telescope'' (VLT), y en los próximos años el ``European Extremely
Large Telescope'' (E-ELT), con el cual se estima que el $60\%$ de la observación
astronómica mundial se realice en Chile.

Una de las condiciones que se establecen a nivel país, es que el $10\%$ del tiempo
de observación pertenece a la comunidad astronómica chilena, lo cual justifica
la necesidad a nivel país del desarrollo de una plataforma astroinformática
para una inteligente administración y análisis.

La necesidad de un sistema con éstas características no es algo nuevo,
debido que desde el $2002$ se planteó éste tipo de problemática, lo cual sugirió
la creación de un Observatoro Virtual (VO, por sus siglas en inglés) como una solución.

El VO es una iniciativa internacional que permite el acceso de datos astronómicos,
a cargo de centros especializados para su almacenamiento y procesamiento,
a los cuales pueden acceder tanto astrónomos, como personas comunes.
Con la estandarización de métodos e información es posible estudiar los registros
astronómicos sin requerimientos físicos de instrumentos y localización.

\subsection{¿Por qué es nuevo?}
VO con datos de ALMA.

Resumen de lo importante del paper.

%\subsection{IVOA Standards}


\subsection{ALMA + Cubos + Formatos propios}
